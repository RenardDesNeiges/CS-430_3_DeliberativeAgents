\documentclass[11pt]{article}

\usepackage{sectsty}
\usepackage{graphicx}
\usepackage[T1]{fontenc}
\usepackage{epigraph}
\usepackage{amssymb}
\usepackage{mathtools}


\usepackage[ruled,vlined]{algorithm2e}
\usepackage{amsthm}


\newtheorem{theorem}{Theorem}[section]
\newtheorem{corollary}{Corollary}[theorem]
\newtheorem{lemma}[theorem]{Lemma}
\newtheorem{problem}{Problem}[section]
\newtheorem{definition}{Definition}[section]
\newtheorem{proposition}{Proposition}[section]

%% declaring abs so that it works nicely
\DeclarePairedDelimiter\abs{\lvert}{\rvert}%
\DeclarePairedDelimiter\norm{\lVert}{\rVert}%

% Swap the definition of \abs* and \norm*, so that \abs
% and \norm resizes the size of the brackets, and the 
% starred version does not.
\makeatletter
\let\oldabs\abs
\def\abs{\@ifstar{\oldabs}{\oldabs*}}
%
\let\oldnorm\norm
\def\norm{\@ifstar{\oldnorm}{\oldnorm*}}
\makeatother

% Marges
\topmargin=-0.45in
\evensidemargin=0in
\oddsidemargin=0in
\textwidth=5.5in
\textheight=9.0in
\headsep=0.5in


\title{CS-430 : Problem Definition for the Reactive Agent Assignment}
\date{\today}
\author{Titouan Renard, Christophe Marciot}

\begin{document}
\maketitle	

\section{Problem definition}

We are trying to implement an agent that functions according to the following basic procedure.

\begin{algorithm}
    \SetAlgoLined
    \caption{Basic Agent Logic}
    \While(){goal not reached}{
        \If(){current plan not applicable anymore}{
            Compute optimal plan
        }
        \Else{
            Execute next action in the plan
        }
    }
\end{algorithm}

In this case optimality is defined as \emph{minimum cost function}. And the cost function is computed for any road taken as such :

\[c(\text{road}) = l_{road} \cdot c_{kil}\]

Where the value $l_{road}$ is the length of a given road and the value $c_{kil}$ is the cost per kilometer for a given agent. \\

Computing an optimal plan can be though of as finding a plan $p_{opt}$ such that:

\[p_{opt} = argmin_{p} \{ cost(p) | \text{ p reaches a goal} \}\]

Where the cost of a plan is simply given by :

\[cost(p) = \sum_{r \in \text{ roads in the plan}} cost(r)\]

The \emph{goal} of our agent is to deliver all tasks on the map, all of these tasks as well as the length of every road and the cost per kilometer are known prior to planning. \\

Our agent has the ability to perform multiple tasks at the same time but has a maximum weight that it can carry, it has to verify : 

\[\sum \text{carried tasks } \leq \text{ agent capacity}\]

We will use \emph{BFS} and \emph{ASTAR} to find an optimal solution.

\section{States, Transitions and Goals}

Since we are working in a deliberative agent paradigm, we will not generate every possible state prior to planning, instead we will define states as a set of variables that are defined by our algorithm as it searches the solution space.

\subsection{States}

The states the agent can be in are defined by the following variables, for a given agent \textbf{agent}:
\begin{enumerate}
    \item \textbf{agent.position} : the city a given agent is in
    \item \textbf{agent.ctasks} : the list of all tasks carried by the agent in a given state
    \item \textbf{agent.cweight} : the total weight carried by the agent in a given state
    \item \textbf{agent.free\_tasks} : the complete list of all undelivered, not yet picked-up tasks at a given state
\end{enumerate}

\subsection{Transitions}

Our agent can perform three basic actions 
\begin{enumerate}
    \item It can drive from one city to another
    \item It can decide to pickup or not to pickup a task when it is in a given city
    \item It can decide to deliver a task when it is in the city it should be delivered to 
\end{enumerate}
We therefore construct our state representation around those three fundamental transitions :
\begin{enumerate}
    \item \textbf{drive(start, destination)} : the movement action between two \emph{adjacent} cities.
    \subitem \emph{Cost function} defined as : $cost_{drive}(road_{start,destination}) = l_{road} \cdot c_{kil} $
    \subitem \emph{Possible} $\iff$ the agent is in the city \emph{start} and the city \emph{destination} is adjacent to \emph{start}.
    \subitem \emph{Consequences} : agent is now in the city \emph{destination}
    \item \textbf{pickup(task)} : the action of adding a task to be carried by the agent
    \subitem \emph{Cost function} defined as : $cost_{pickup}=0$
    \subitem \emph{Possible} $\iff$ the agent is in the city where \emph{task} is and if \[\textit{agent.cweight}+\text{ w$_{task}$} \leq \text{ capacity}\]
    \subitem \emph{Consequences} : agent appends \emph{task} to \emph{agent.ctasks} and adds the weight $w_{task}$ of the new task to \emph{agent.cweight}, \emph{task} is removed from \emph{agent.free\_tasks}
    \item \textbf{deliver(task)} : the action of delivering a task carried by the agent to it's destination
    \subitem \emph{Cost function} defined as : $cost_{deliver}=0$
    \subitem \emph{Possible} $\iff$ the agent is in the city where \emph{task} should be delivered
    \subitem \emph{Consequences} : agent removes \emph{task} from \emph{agent.ctasks} and substracts the weight $w_{task}$ from the task to \emph{agent.cweight}
\end{enumerate}

\subsection{Goal}

The \emph{goal} state is defined as any state where :
\[\text{agent.free\_tasks } = \emptyset\]

\section{Implementation of the re-planning algorithm}

\textbf{Problem: }if every agent has it's own \emph{agent.free\_tasks} data-structure, it will not be able to cope with the fact that some other agent may take a task and remove it from the actual list of free tasks.\\
\textbf{Solution: }check at each state wether or not the free tasks have been modified, if it is the case, replan accordingly. This is essentially what the \emph{Basic agent logic} algorithm suggests but this is a somewhat more detailed implementation:

\begin{algorithm}
    \SetAlgoLined
    \caption{Replanning Implementation}
    \textbf{Initialization :} $p_{opt}\leftarrow $ compute\_optimal\_plan(\emph{global.free\_tasks})  \\
    \While(){goal not reached}{
        \If(){\emph{agent.free\_task} $\neq$ \emph{global.free\_tasks}}{
            $p_{opt} \leftarrow$ compute\_optimal\_plan(\emph{global.free\_tasks}) \\
        }
        execute next action in $p_{opt}$
    }
\end{algorithm}

\section{ASTAR Heuristics Definition}
?????

\end{document}

